\section{Project Design}

\subsection{设计}
% 系统功能、端口规格、开发板上使用到的输入输出设备
本次项目几乎实现了文档中所有描述的功能:
\vspace{-1.3em}
\paragraph{\ding{172} 进场模式} 闲置状态下,系统滚动显示剩余车位等信息,并允许用户在进场时选择停入A区或B区。若无剩余车位则拒绝停入,否则将占用一个车位并返回标识码。
\paragraph{\ding{173} 出场模式} 

\subsection{Structure Design}
% 各模块接口及功能、状态迁移/事务处理流程图

\subsubsection{主状态机}
\paragraph{接口和变量描述} 主状态机接收6个输入,分别为 时钟信号 clk2 (1Hz), clk3 (500Hz),使能信号 enable,5-bit按钮 button,30-bit键盘输入 key,键盘输入结束信号 finish。其输出16个信号,分别为显示模式 displayMode,剩余车位 left,起始价格 st,时价 per, VIP用户ID {id1,id0},VIP余额 {remain1,remain0},以及七段数码显示管 {x7,x6,x5,x4,x3,x2,x1,x0}。
\par 状态机现态和次态使用6位寄存器储存,并且将每个状态设为参数,便于理解。主要状态如下所示:
\begin{lstlisting}
parameter Idle = 6'b000001;
parameter InputVip = 6'b000010;
parameter Register = 6'b000011;
parameter VipRecharge = 6'b000100;
parameter VipInterface = 6'b000101;
\end{lstlisting}



\subsection{Verilog Designs \& Port Constraints}
% 核心代码及说明/就主要的输入输出端口分开进行描述

\subsubsection{Music player}

The music player, basically, is a FSM that 